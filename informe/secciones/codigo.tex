\subsection{Flex}
\lstinputlisting[basicstyle=\fontsize{7}{9}\selectfont\ttfamily,breaklines=true,language=C]{src/tokens.l}

Definimos todos los tokens posibles seg�n el lenguaje MyLanga. Cabe destacar que para hacer comentarios multil�nea se hace uso de un cambio de contexto en Flex que solo puede retornar al contexto original una vez que lee el token de fin de comentario.

\newpage
\subsection{Bison}

\lstinputlisting[basicstyle=\fontsize{7}{9}\selectfont\ttfamily,breaklines=true,language=C]{src/parser.y}

Las reglas de bison son muy similares a las de la gram�tica definida anteriormente. La principal diferencia es que se agregan reglas de error para que matcheen expl�citamente algunos casos de error espec�ficos y de esa manera poder devolver mensajes m�s declarativos.
A su vez se ve como se arma el �rbol \emph{AST} que luego ser� utilizado para la ejecuci�n del c�digo. Es decir, al programa se le agrega una instancia de su bloque de funciones y su llamado a plot. Al bloque de funciones se le va empujando cada funci�n definida. Dentro de cada funci�n se agrega cada instrucci�n, etc.

\newpage
\subsection{Implementaci�n del AST}
\subsubsection{Definici�n de clases y estructuras}
Primero definimos todas las clases que se usaron para implementar el AST. Se puede ver que son las mismas que se llaman desde las reglas de Bison.
\newline

\lstinputlisting[basicstyle=\fontsize{7}{9}\selectfont\ttfamily,breaklines=true,language=C]{src/mylanga_ast.h}
\newpage
\subsubsection{Implementaci�n de clases y estructuras}
Finalmente, su implementaci�n.
\newline

\lstinputlisting[basicstyle=\fontsize{7}{9}\selectfont\ttfamily,breaklines=true,language=C]{src/mylanga_ast.cpp}