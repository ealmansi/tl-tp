\newpage
\section{Resoluci�n y aclaraciones}
\subsection{Generalidades}

Para la resoluci�n del problema planteado se decidi� utilizar dos de las herramientas propuestas en clase. El analizador l�xico $\bold{Flex}$ se utiliz� para especificar los tokens del lenguaje MyLanga, generando un \emph{scanner} que transforma el c�digo a procesar en un stream de tokens. Estos tokens pasan a ser los terminales de la gram�tica que se utiliza para describir con precisi�n el lenguaje, a partir de la cual el generador de parsers $\bold{Bison}$ construye un parser para MyLanga.

La definici�n de la gram�tica que se provee como entrada a $\bold{Bison}$, asociando a cada producci�n una regla sem�ntica, la cual genera la l�gica necesaria para construir el �rbol de sintaxis, o \emph{AST}, que representa el programa procesado.

De esta forma, el int�rprete de MyLanga desarrollado utiliza el parser generado de forma autom�tica, y posteriormente ejecuta el c�digo codificado en el \emph{AST}.
\newline

Algunas aclaraciones de la implementaci�n:

\begin{itemize}
  \item Adem�s de los nodos del �rbol que representan un grupo de funciones, una instrucci�n, un if, etc, hay un objeto global que contiene la informaci�n contextual del programa, cumpliendo un prop�sito doble: por un lado, contiene un diccionario con las funciones definidas en el programa, y por el otro, se encarga de generar los scopes durante la ejecuci�n. El mismo se comporta como un stack y se utiliza para saber que variables son visibles tanto a la hora de parsear como ejecutar el c�digo. Cuando se realza un llamado a funci�n, se pushea un estado nuevo al stack dejando vac�a la definici�n de variables (excepto por los par�metros si hay alguno). Una vez que se retorna de la funci�n, se popea el stack para volver al estado inicial del llamador.
  \item Una vez generado el \emph{AST}, se realiza un proceso de verificaci�n de validez sobre el programa. El mismo consiste en recorrer recursivamente el �rbol, evaluando en cada nodo si este es v�lido, as� como tambi�n sus hijos. El procedimiento se propaga independientemente de si se encuentra efectivamente alg�n error, permitiendo detectar m�ltiples problemas en una �nica pasada. Posteriormente, bajo la condici�n de que todos los nodos del �rbol verificaran su condici�n de validez, se procede a ejecutar el comado plot.
  \item Se agreg� la compatibilidad con c�digo que hace recursi�n expl�cita. Esto se logr� definiendo una funci�n inmediatamente luego de leer su signatura, previo a evaluar la validez de su bloque de c�digo correspondiente. Naturalmente, si se encuentra alg�n error en el cuerpo de la funci�n, se elimina la misma de la tabla de contexto, generando posiblemente una serie de errores en el resto del programa donde la misma sea utilizada.
\end{itemize}