\newpage
\section{Resoluci�n y aclaraciones}
\subsection{Generalidades}

Para la resoluci�n del problema planteado se decidi� utilizar dos de las herramientas propuestas en clase. $\bold{Flex}$ para generar el analizador l�xico que provea los tokens del c�digo MyLanga, y $\bold{Bison}$ como generador del parser que recibe dichos tokens.
Desde el c�digo bison se agrega adem�s la l�gica para construir el \emph{AST} que luego se ejecutar� recursivamente.
\newline

Algunas aclaraciones de la implementaci�n:

\begin{itemize}
  \item Adem�s de los nodos del �rbol que representan un grupo de funciones, una instrucci�n, un if, etc, hay un objeto global que se utiliza para generar scopes. El mismo se comporta como un stack y se utiliza para saber que variables son visibles tanto a la hora de parsear como ejecutar el c�digo. Cuando se llama a una funci�n, se pushea un estado nuevo al stack dejando vac�a la definici�n de variables (excepto por los par�metros si hay alguno). Una vez que se retorna de la funci�n, se popea el stack para volver al estado inicial del llamador.
  \item Agregamos la funcionalidad de escribir sentencias directamente en el comando plot. Es decir, en vez de definir por ejemplo una funci�n que haga \emph{x+x}, podemos escribirlo directamente en plot. Esto en particular hace que algunos casos que en los tests propuestos debieran dar un error, en nuestro compilador sean aceptados como v�lidos.
  \item Se agreg� la compatibilidad con c�digo que hace recursi�n expl�cita.
\end{itemize}