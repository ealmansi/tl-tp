\newpage
\section{Introduccion}

\noindent Se debe programar un compilador que parsee y ejecute c�digo perteneciente al lenguaje \newline ``MyLanga''.
Un programa bien formado en dicho lenguaje es una secuencia de definiciones de funciones, seguida de una �ltima sentencia de ploteo. Dicha sentencia define un rango similar a un \emph{for} de una variable (o de un rango de Smalltalk, ya que define un desde-hasta y un step para cada iteraci�n) y toma dos funciones como par�metro que son las que generan un x,y en cada iteraci�n. Estos puntos son devueltos para graficarse mediante gnuplot.

El lenguaje tiene una sintaxis similar a \emph{C}. En particular los operadores l�gicos y aritm�ticos se pueden considerar id�nticos. Al igual que \emph{while} e \emph{if} y sus guardas. Las diferencias m�s notables con respecto a \emph{C} ser�an:

\begin{itemize}
  \item Falta de $\bold{;}$ al final de cada instrucci�n.
  \item Posibilidad de definir funciones sin llaves para el cuerpo cuando el mismo es de una sola instrucci�n.
  \item Cuando el cuerpo del if tiene una sola instrucci�n se agrega el keyword $\bold{then}$ si se desea evitar llaves.
	\item Constante $\bold{pi}$.
	\item Las variables no se declaran ni se especifica est�ticamente su tipo.
\end{itemize}

Otro problema a tener en cuenta es la detecci�n de errores y la devoluci�n de mensajes declarativos cuando los hubiera.