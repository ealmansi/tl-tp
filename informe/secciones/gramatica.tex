\subsection{Gram�tica}

Definimos la siguiente gram�tica G = $<$ \{ Program, Seq\_fun\_def, Fun\_def, Lst\_id, Lst\_id2, Block, Seq\_stmt, Stmt, Pred, Expr, Lst\_expr, Lst\_expr2, Plot\_cmd \}, \{ for, plot, if, then, else, while, return, function, pi, id, int, float, $+$, $-$, $*$, $/$, $\wedge$, $\;$, ,, .., $=$, $||$, $\&\&$, !, $<$, $<=$, $==$, $>=$, $>$, $($, $)$, $\{$, $\}$ \}, P, Program $>$ donde P son las siguientes producciones:
\newline

\noindent Program $\rightarrow$ Seq\_fun\_def Plot\_cmd 
\newline
\noindent \emph{Una programa v�lido es una secuencia de funciones y un comando plot.}
\newline

\noindent Seq\_fun\_def $\rightarrow$ Fun\_def $|$ Seq\_fun\_def Fun\_def 
\newline
\noindent \emph{Una secuencia de funciones es una funci�n o una secuencia seguida de una funci�n.}
\newline

\noindent Fun\_def $\rightarrow$ function id ( Lst\_id ) Block
\newline
\noindent \emph{Una funci�n bien definida es un ID de funci�n seguido de los par�metros (lista de ids) rodeados por par�ntesis y por �ltimo un bloque de instrucciones.}
\newline

\noindent Lst\_id $\rightarrow$ $\lambda$ $|$ Lst\_id2
\newline
\noindent \emph{Una lista de par�metros es o bien una lista vac�a o una con par�metros.}
\newline

\noindent Lst\_id2 $\rightarrow$ id $|$ Lst\_id2 , id
\newline
\noindent \emph{Una lista con par�metros es o un ID o una lista seguida de una coma y otro ID.}
\newline

\noindent Block $\rightarrow$ Stmt $|$ \{ Seq\_stmt \}
\newline
\noindent \emph{Un bloque de instrucciones es o bien una �nica instrucci�n o una secuencia de instrucciones entre llaves.}
\newline

\noindent Seq\_stmt $\rightarrow$ Stmt $|$ Seq\_stmt Stmt 
\newline
\noindent \emph{Una secuencia de instrucciones es o bien una instrucci�n o una secuencia seguida de una instrucci�n.}
\newline

\noindent Stmt $\rightarrow$ id = Expr \newline
$|$ if Pred then Block \newline
$|$ if Pred then Block else Block \newline
$|$ while Pred block \newline
$|$ return Expr \newline
\noindent \emph{Las instrucciones son o bien una asignaci�n, un if (con y sin else), un while o un return.}
\newline

\noindent Pred $\rightarrow$ Expr $<$ Expr \newline
$|$ Expr $<=$ Expr \newline
$|$ Expr $==$ Expr \newline
$|$ Expr $>=$ Expr \newline
$|$ Expr $>$ Expr \newline
$|$ Pred $||$ Pred \newline
$|$ Pred \&\& Pred \newline
$|$ ! Pred \newline
$|$ ( Pred ) \newline
\noindent \emph{Los predicados se componen mediante operadores binarios o negaciones. Tambi�n pueden estar rodeados por par�ntesis.}
\newpage

\noindent Expr $\rightarrow$ int \newline
$|$ float \newline
$|$ pi \newline
$|$ id \newline
$|$ Expr + Expr \newline
$|$ Expr - Expr \newline
$|$ Expr * Expr \newline
$|$ Expr / Expr \newline
$|$ Expr $\wedge$ Expr \newline
$|$ - Expr \newline
$|$ id ( Lst\_expr ) \newline
$|$ ( Expr ) \newline
\noindent \emph{Las expresiones son o bien literales de enteros y flotantes, o un id de variable, o una operaci�n aritm�tica entre expresiones. Tambi�n pueden ser llamados a funciones, la negativizaci�n, o la constante pi. Por �ltimo, tambi�n pueden estar rodeadas por par�ntesis.}
\newline

\noindent Lst\_expr $\rightarrow$ $\lambda$ $|$ Lst\_expr2
\newline
\noindent \emph{Una lista de expresiones es o bien una lista vac�a o una con expresiones.}
\newline

\noindent Lst\_expr2 $\rightarrow$ Expr $|$ Lst\_expr2 , Expr
\newline
\noindent \emph{Una lista con expresiones es o una expresi�n o una lista seguida de una coma y otra expresi�n.}
\newline

\noindent Plot\_cmd $\rightarrow$ plot ( Expr , Expr ) for id $=$ Expr .. Expr .. Expr
\newline
\noindent \emph{Por �ltimo el comando plot recibe dos expresiones a llamar y un for con 3 expresiones con un valor, un tope y un incremento para una variable id.}
\newline